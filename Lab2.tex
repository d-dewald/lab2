% Options for packages loaded elsewhere
\PassOptionsToPackage{unicode}{hyperref}
\PassOptionsToPackage{hyphens}{url}
%
\documentclass[
]{article}
\usepackage{lmodern}
\usepackage{amssymb,amsmath}
\usepackage{ifxetex,ifluatex}
\ifnum 0\ifxetex 1\fi\ifluatex 1\fi=0 % if pdftex
  \usepackage[T1]{fontenc}
  \usepackage[utf8]{inputenc}
  \usepackage{textcomp} % provide euro and other symbols
\else % if luatex or xetex
  \usepackage{unicode-math}
  \defaultfontfeatures{Scale=MatchLowercase}
  \defaultfontfeatures[\rmfamily]{Ligatures=TeX,Scale=1}
\fi
% Use upquote if available, for straight quotes in verbatim environments
\IfFileExists{upquote.sty}{\usepackage{upquote}}{}
\IfFileExists{microtype.sty}{% use microtype if available
  \usepackage[]{microtype}
  \UseMicrotypeSet[protrusion]{basicmath} % disable protrusion for tt fonts
}{}
\makeatletter
\@ifundefined{KOMAClassName}{% if non-KOMA class
  \IfFileExists{parskip.sty}{%
    \usepackage{parskip}
  }{% else
    \setlength{\parindent}{0pt}
    \setlength{\parskip}{6pt plus 2pt minus 1pt}}
}{% if KOMA class
  \KOMAoptions{parskip=half}}
\makeatother
\usepackage{xcolor}
\IfFileExists{xurl.sty}{\usepackage{xurl}}{} % add URL line breaks if available
\IfFileExists{bookmark.sty}{\usepackage{bookmark}}{\usepackage{hyperref}}
\hypersetup{
  pdftitle={Lab 2},
  pdfauthor={Diana DeWald},
  hidelinks,
  pdfcreator={LaTeX via pandoc}}
\urlstyle{same} % disable monospaced font for URLs
\usepackage[margin=1in]{geometry}
\usepackage{color}
\usepackage{fancyvrb}
\newcommand{\VerbBar}{|}
\newcommand{\VERB}{\Verb[commandchars=\\\{\}]}
\DefineVerbatimEnvironment{Highlighting}{Verbatim}{commandchars=\\\{\}}
% Add ',fontsize=\small' for more characters per line
\usepackage{framed}
\definecolor{shadecolor}{RGB}{248,248,248}
\newenvironment{Shaded}{\begin{snugshade}}{\end{snugshade}}
\newcommand{\AlertTok}[1]{\textcolor[rgb]{0.94,0.16,0.16}{#1}}
\newcommand{\AnnotationTok}[1]{\textcolor[rgb]{0.56,0.35,0.01}{\textbf{\textit{#1}}}}
\newcommand{\AttributeTok}[1]{\textcolor[rgb]{0.77,0.63,0.00}{#1}}
\newcommand{\BaseNTok}[1]{\textcolor[rgb]{0.00,0.00,0.81}{#1}}
\newcommand{\BuiltInTok}[1]{#1}
\newcommand{\CharTok}[1]{\textcolor[rgb]{0.31,0.60,0.02}{#1}}
\newcommand{\CommentTok}[1]{\textcolor[rgb]{0.56,0.35,0.01}{\textit{#1}}}
\newcommand{\CommentVarTok}[1]{\textcolor[rgb]{0.56,0.35,0.01}{\textbf{\textit{#1}}}}
\newcommand{\ConstantTok}[1]{\textcolor[rgb]{0.00,0.00,0.00}{#1}}
\newcommand{\ControlFlowTok}[1]{\textcolor[rgb]{0.13,0.29,0.53}{\textbf{#1}}}
\newcommand{\DataTypeTok}[1]{\textcolor[rgb]{0.13,0.29,0.53}{#1}}
\newcommand{\DecValTok}[1]{\textcolor[rgb]{0.00,0.00,0.81}{#1}}
\newcommand{\DocumentationTok}[1]{\textcolor[rgb]{0.56,0.35,0.01}{\textbf{\textit{#1}}}}
\newcommand{\ErrorTok}[1]{\textcolor[rgb]{0.64,0.00,0.00}{\textbf{#1}}}
\newcommand{\ExtensionTok}[1]{#1}
\newcommand{\FloatTok}[1]{\textcolor[rgb]{0.00,0.00,0.81}{#1}}
\newcommand{\FunctionTok}[1]{\textcolor[rgb]{0.00,0.00,0.00}{#1}}
\newcommand{\ImportTok}[1]{#1}
\newcommand{\InformationTok}[1]{\textcolor[rgb]{0.56,0.35,0.01}{\textbf{\textit{#1}}}}
\newcommand{\KeywordTok}[1]{\textcolor[rgb]{0.13,0.29,0.53}{\textbf{#1}}}
\newcommand{\NormalTok}[1]{#1}
\newcommand{\OperatorTok}[1]{\textcolor[rgb]{0.81,0.36,0.00}{\textbf{#1}}}
\newcommand{\OtherTok}[1]{\textcolor[rgb]{0.56,0.35,0.01}{#1}}
\newcommand{\PreprocessorTok}[1]{\textcolor[rgb]{0.56,0.35,0.01}{\textit{#1}}}
\newcommand{\RegionMarkerTok}[1]{#1}
\newcommand{\SpecialCharTok}[1]{\textcolor[rgb]{0.00,0.00,0.00}{#1}}
\newcommand{\SpecialStringTok}[1]{\textcolor[rgb]{0.31,0.60,0.02}{#1}}
\newcommand{\StringTok}[1]{\textcolor[rgb]{0.31,0.60,0.02}{#1}}
\newcommand{\VariableTok}[1]{\textcolor[rgb]{0.00,0.00,0.00}{#1}}
\newcommand{\VerbatimStringTok}[1]{\textcolor[rgb]{0.31,0.60,0.02}{#1}}
\newcommand{\WarningTok}[1]{\textcolor[rgb]{0.56,0.35,0.01}{\textbf{\textit{#1}}}}
\usepackage{graphicx,grffile}
\makeatletter
\def\maxwidth{\ifdim\Gin@nat@width>\linewidth\linewidth\else\Gin@nat@width\fi}
\def\maxheight{\ifdim\Gin@nat@height>\textheight\textheight\else\Gin@nat@height\fi}
\makeatother
% Scale images if necessary, so that they will not overflow the page
% margins by default, and it is still possible to overwrite the defaults
% using explicit options in \includegraphics[width, height, ...]{}
\setkeys{Gin}{width=\maxwidth,height=\maxheight,keepaspectratio}
% Set default figure placement to htbp
\makeatletter
\def\fps@figure{htbp}
\makeatother
\setlength{\emergencystretch}{3em} % prevent overfull lines
\providecommand{\tightlist}{%
  \setlength{\itemsep}{0pt}\setlength{\parskip}{0pt}}
\setcounter{secnumdepth}{-\maxdimen} % remove section numbering

\title{Lab 2}
\author{Diana DeWald}
\date{10/11/2020}

\begin{document}
\maketitle

Run the following code to (a) install the \{nlme\} and \{janitor\}
packages (b) load the packages along with the tidyverse, and (c) access
and quickly prep some data (from the \{nmle\} package) for plotting.

\begin{Shaded}
\begin{Highlighting}[]
\CommentTok{# Note: You only need to run this next line one time to install these two packages.}
\KeywordTok{install.packages}\NormalTok{(}\KeywordTok{c}\NormalTok{(}\StringTok{"nlme"}\NormalTok{, }\StringTok{"janitor"}\NormalTok{)) }\CommentTok{#(a)}
\end{Highlighting}
\end{Shaded}

\begin{Shaded}
\begin{Highlighting}[]
\KeywordTok{library}\NormalTok{(nlme) }\CommentTok{# (b)}
\KeywordTok{library}\NormalTok{(janitor) }\CommentTok{# (b)}
\KeywordTok{library}\NormalTok{(tidyverse) }\CommentTok{# (b)}
\KeywordTok{theme_set}\NormalTok{(}\KeywordTok{theme_minimal}\NormalTok{()) }\CommentTok{# (b)}

\NormalTok{pd <-}\StringTok{ }\NormalTok{Oxboys }\OperatorTok\StringTok{ }\CommentTok{# (c)}
\StringTok{  }\KeywordTok{clean_names}\NormalTok{() }\OperatorTok\StringTok{ }
\StringTok{  }\KeywordTok{mutate}\NormalTok{(}\DataTypeTok{subject =} \KeywordTok{factor}\NormalTok{(subject),}
         \DataTypeTok{occasion =} \KeywordTok{factor}\NormalTok{(occasion)) }\OperatorTok\StringTok{ }
\StringTok{  }\KeywordTok{filter}\NormalTok{(subject }\OperatorTok{==}\StringTok{ "10"} \OperatorTok{|}\StringTok{ }\NormalTok{subject }\OperatorTok{==}\StringTok{ "4"} \OperatorTok{|}\StringTok{ }\NormalTok{subject }\OperatorTok{==}\StringTok{ "7"}\NormalTok{) }\OperatorTok\StringTok{ }
\StringTok{  }\KeywordTok{tbl_df}\NormalTok{()}
\end{Highlighting}
\end{Shaded}

\begin{enumerate}
\def\labelenumi{\arabic{enumi}.}
\tightlist
\item
  Reproduce the following two plots, using the \emph{pd} data. You can
  use whatever theme you want (I used \texttt{theme\_minimal()}), but
  all else should be the same.
\end{enumerate}

\begin{Shaded}
\begin{Highlighting}[]
\CommentTok{# Put code below for Plot 1. Note that Plot 1 is a line plot, not a smooth.}

\KeywordTok{ggplot}\NormalTok{(pd, }\KeywordTok{aes}\NormalTok{(age, height)) }\OperatorTok{+}
\StringTok{  }\KeywordTok{geom_line}\NormalTok{() }\OperatorTok{+}
\StringTok{  }\KeywordTok{facet_wrap}\NormalTok{(}\OperatorTok{~}\NormalTok{subject) }\OperatorTok{+}
\StringTok{  }\KeywordTok{labs}\NormalTok{(}\DataTypeTok{title =} \StringTok{"Plot 1"}\NormalTok{)}
\end{Highlighting}
\end{Shaded}

\includegraphics{Lab2_files/figure-latex/plots1-1.pdf}

\begin{Shaded}
\begin{Highlighting}[]
\CommentTok{# Put code below for Plot 2. Note that Plot 2 is a line plot also.}
\KeywordTok{ggplot}\NormalTok{(pd, }\KeywordTok{aes}\NormalTok{(age, height, }\DataTypeTok{color =}\NormalTok{ subject)) }\OperatorTok{+}
\StringTok{  }\KeywordTok{geom_line}\NormalTok{() }\OperatorTok{+}
\StringTok{  }\KeywordTok{labs}\NormalTok{(}\DataTypeTok{title =} \StringTok{"Plot 2"}\NormalTok{)}
\end{Highlighting}
\end{Shaded}

\includegraphics{Lab2_files/figure-latex/plots1-2.pdf}

\newpage

\begin{enumerate}
\def\labelenumi{\arabic{enumi}.}
\setcounter{enumi}{1}
\tightlist
\item
  Use the \emph{mtcars} dataset from base R to replicate the following
  plots. (Just type \emph{mtcars} into the console to see the dataset).
\end{enumerate}

\begin{Shaded}
\begin{Highlighting}[]
\CommentTok{# Put code below for Plot 3}
\KeywordTok{ggplot}\NormalTok{(mtcars, }\KeywordTok{aes}\NormalTok{(drat, mpg)) }\OperatorTok{+}
\StringTok{  }\KeywordTok{geom_point}\NormalTok{() }\OperatorTok{+}
\StringTok{  }\KeywordTok{labs}\NormalTok{(}\DataTypeTok{title =} \StringTok{"Plot 3"}\NormalTok{)}
\end{Highlighting}
\end{Shaded}

\includegraphics{Lab2_files/figure-latex/plots2-1.pdf}

\begin{Shaded}
\begin{Highlighting}[]
\CommentTok{# Put code below for Plot 4}
\KeywordTok{ggplot}\NormalTok{(mtcars, }\KeywordTok{aes}\NormalTok{(drat, mpg)) }\OperatorTok{+}
\StringTok{  }\KeywordTok{geom_point}\NormalTok{() }\OperatorTok{+}
\StringTok{  }\KeywordTok{geom_smooth}\NormalTok{() }\OperatorTok{+}
\StringTok{  }\KeywordTok{labs}\NormalTok{(}\DataTypeTok{title =} \StringTok{"Plot 4"}\NormalTok{)}
\end{Highlighting}
\end{Shaded}

\includegraphics{Lab2_files/figure-latex/plots2-2.pdf}

\begin{Shaded}
\begin{Highlighting}[]
\CommentTok{# Put code below for Plot 5}
\KeywordTok{ggplot}\NormalTok{(mtcars, }\KeywordTok{aes}\NormalTok{(drat, mpg)) }\OperatorTok{+}
\StringTok{  }\KeywordTok{geom_point}\NormalTok{() }\OperatorTok{+}
\StringTok{  }\KeywordTok{geom_smooth}\NormalTok{(}\DataTypeTok{method =} \StringTok{"lm"}\NormalTok{) }\OperatorTok{+}
\StringTok{  }\KeywordTok{facet_wrap}\NormalTok{(}\OperatorTok{~}\NormalTok{vs) }\OperatorTok{+}
\StringTok{  }\KeywordTok{labs}\NormalTok{(}\DataTypeTok{title =} \StringTok{"Plot 5"}\NormalTok{)}
\end{Highlighting}
\end{Shaded}

\includegraphics{Lab2_files/figure-latex/plots2-3.pdf}

\begin{Shaded}
\begin{Highlighting}[]
\CommentTok{# Put code below for Plot 6. I have given you the first line of code to start.}
\KeywordTok{ggplot}\NormalTok{(mtcars, }\KeywordTok{aes}\NormalTok{(drat, mpg, }\DataTypeTok{color =} \KeywordTok{factor}\NormalTok{(cyl))) }\OperatorTok{+}
\StringTok{  }\KeywordTok{geom_point}\NormalTok{() }\OperatorTok{+}
\StringTok{  }\KeywordTok{geom_smooth}\NormalTok{(}\DataTypeTok{method =} \StringTok{"lm"}\NormalTok{, }\DataTypeTok{se =} \OtherTok{FALSE}\NormalTok{) }\OperatorTok{+}
\StringTok{  }\KeywordTok{labs}\NormalTok{(}\DataTypeTok{title =} \StringTok{"Plot 6"}\NormalTok{)}
\end{Highlighting}
\end{Shaded}

\includegraphics{Lab2_files/figure-latex/plots2-4.pdf}

\newpage

\begin{enumerate}
\def\labelenumi{\arabic{enumi}.}
\setcounter{enumi}{2}
\tightlist
\item
  Use the \emph{Orange} dataset, also part of base R, to replicate the
  following plots.
\end{enumerate}

\begin{Shaded}
\begin{Highlighting}[]
\CommentTok{# Put code below for Plot 7}
\KeywordTok{ggplot}\NormalTok{(Orange, }\KeywordTok{aes}\NormalTok{(age, circumference, }\DataTypeTok{color =}\NormalTok{ Tree)) }\OperatorTok{+}
\StringTok{  }\KeywordTok{geom_point}\NormalTok{() }\OperatorTok{+}
\StringTok{  }\KeywordTok{geom_line}\NormalTok{() }\OperatorTok{+}
\StringTok{  }\KeywordTok{labs}\NormalTok{(}\DataTypeTok{title =} \StringTok{"Plot 7"}\NormalTok{)}
\end{Highlighting}
\end{Shaded}

\includegraphics{Lab2_files/figure-latex/orange-plots-1.pdf}

\begin{Shaded}
\begin{Highlighting}[]
\CommentTok{# Put code below for the last plot. See slide 51 from the w2p2 class for labels.}
\KeywordTok{ggplot}\NormalTok{(Orange, }\KeywordTok{aes}\NormalTok{(age, circumference, }\DataTypeTok{color =}\NormalTok{ Tree)) }\OperatorTok{+}
\StringTok{  }\KeywordTok{geom_point}\NormalTok{(}\DataTypeTok{cex =} \DecValTok{3}\NormalTok{) }\OperatorTok{+}
\StringTok{  }\KeywordTok{geom_smooth}\NormalTok{(}\DataTypeTok{method =} \StringTok{"lm"}\NormalTok{, }\DataTypeTok{se =} \OtherTok{FALSE}\NormalTok{, }\DataTypeTok{color =} \StringTok{"gray42"}\NormalTok{) }\OperatorTok{+}
\StringTok{  }\KeywordTok{labs}\NormalTok{(}\DataTypeTok{x =} \StringTok{"Age of the Tree (in days)"}\NormalTok{,}
       \DataTypeTok{y =} \StringTok{"Circumference of the Trunk (in mm)"}\NormalTok{,}
       \DataTypeTok{title =} \StringTok{"Orange Tree Growth"}\NormalTok{,}
       \DataTypeTok{subtitle =} \StringTok{"Gray line displays a linear model fit to the data."}\NormalTok{)}
\end{Highlighting}
\end{Shaded}

\includegraphics{Lab2_files/figure-latex/orange-plots-2.pdf}

\end{document}
